\documentclass[12pt]{article}
\usepackage[top=1in,bottom=1in,left=1in,right=1in]{geometry}
\usepackage{alltt}
\usepackage{array}	
\usepackage{graphicx}
\usepackage{tabularx}
\usepackage{verbatim}
\usepackage{setspace}
\usepackage{listings}
\usepackage{amssymb,amsmath, amsthm}
\usepackage{hyperref}
\usepackage{oz}
\usepackage[cc]{titlepic}
\usepackage{fancyvrb}
\usepackage{datetime}

\title{SOEN 342 - Sections H and II:\\Software Requirements and Specifications\\
	\ \\
	Iteration 2: Formal specifications}
\author{Antoine Cantin \\ 40211205 \and Tuan Anh Pham \\ 40213926  \and Mustafa Sameem \\40190889}

\newdate{mydate}{04}{11}{2023}
\date{\displaydate{mydate}}


\begin{spacing}{1.5}
	\begin{document}
		\maketitle
		
		\newpage

		\section{Complete formal specification in Z}
		
		The formal specification of the system introduces the following three types:
		
		\[ SENSOR\_TYPE, LOCATION\_TYPE, TEMPERATURE\_TYPE  \]
		
		\noindent The system's (partial) formal specification is given in the Z language and it consists of schemas and the definitions of operations that constitute the system's exposed interface.
		
	
		\subsection{Schemas}
		
		
		\begin{schema}{TempMonitor}
			deployed~:~\mathbb{P}~SENSOR\_TYPE\\
			map : SENSOR\_TYPE \nrightarrow LOCATION\_TYPE\\
			read : SENSOR\_TYPE  \nrightarrow TEMPERATURE\_TYPE\\
            sensorRegistry~:~\mathbb{P}~SENSOR\_TYPE\\
            locationRegistry~:~\mathbb{P}~LOCATION\_TYPE
			\where
            deployed \subseteq sensorRegistry\\
			deployed = \dom map\\
			deployed = \dom read
		\end{schema}
		
		
		
		\begin{schema}{DeploySensorOK}
			\Delta TempMonitor\\
			sensor? : SENSOR\_TYPE\\
			location? : LOCATION\_TYPE\\
			temperature? : TEMPERATURE\_TYPE
			\where
			sensor? \notin deployed\\
			location? \notin \ran map\\
			deployed' = deployed \cup \{ sensor? \}\\
			map' = map \cup \{ sensor? \mapsto location? \}\\
			read' = read \cup \{ sensor? \mapsto temperature? \}
		\end{schema}
		
		
		\begin{schema}{ReadTemperatureOK}
			\Xi TempMonitor\\
			location? : LOCATION\_TYPE\\
			temperature! : TEMPERATURE\_TYPE
			\where
			location? \in \ran map\\
			temperature! = read(map^{-1}(location?))\\
		\end{schema}

        \begin{schema}{ReplaceSensorOK}
            \Delta TempMonitor \\
            old~sensor? : SENSOR\_TYPE \\
            new~sensor? : SENSOR\_TYPE \\
            \where
            old~sensor? \in \dom map \\
            old~sensor? \in \dom read \\
            new~sensor? \in \dom sensorRegistry \\
            new~sensor? \notin deployed \\
            location' = map~(old~sensor?) \\
            temperature' = read~(old~sensor?) \\
            deployed' = deployed \cup \{ new~sensor? \} \\
            map' = \{ old~sensor? \} \dsub map \\
            read' = \{ old~sensor? \} \dsub read \\
            deployed' = deployed \setminus \{ old~sensor?\} \\
            sensorRegistry' = sensorRegistry \setminus \{ old~sensor?\} \\
            map'' = map' \cup \{ new~sensor? \mapsto location' \} \\
            read'' = read' \cup \{ new~sensor? \mapsto temperature' \}
        \end{schema}
        


        \begin{schema}{RemoveSensorOK}
            \Delta TempMonitor\\
            sensor? : SENSOR\_TYPE\\
            \where
            \t1 \textit{(\small Case 2: Sensor is not in the deployed subset)} \\
            sensor? \notin deployed \\
            sensorRegistry' = sensorRegistry \setminus \{ sensor?\}\\
        \end{schema}
        \begin{schema}{RemoveDeployedSensorOK}
            \Delta TempMonitor\\
            sensor? : SENSOR\_TYPE\\
            \where
            \t1 \textit{(\small Case 1: Sensor is in the deployed subset)} \\
            sensor? \in deployed \\
            map' = \{ sensor? \} \dsub map \\
            read' = \{ sensor? \} \dsub read \\
            deployed' = deployed \setminus \{ sensor?\}\\
            sensorRegistry' = sensorRegistry \setminus \{ sensor?\}
        \end{schema}



        \begin{schema}{ReturnLocationTemperatureCollection}
         \Xi TempMonitor \\
         temp: LOCATION\_TYPE \pfun TEMPERATURE\_TYPE \\
         output!: \seq (LOCATION\_TYPE \cross TEMPERATURE\_TYPE)
        \where
         temp = \{ l: LOCATION\_TYPE | l \in ran(map) \implies l \mapsto read(map^{-1}(l)) \} \\
         output! = \langle l, t: LOCATION\_TYPE \cross TEMPERATURE\_TYPE | (l \mapsto t) \in \temp \implies (l, t) \rangle \\
        \end{schema}

        
	
		\begin{schema}{Success}
			\Xi TempMonitor\\
			response! : MESSAGE
			\where
			response!~=~'ok'\\
		\end{schema}
		
		
		
		\begin{schema}{SensorAlreadyDeployed}
			\Xi TempMonitor\\
			sensor? : SENSOR\_TYPE\\
			response! : MESSAGE
			\where
			sensor? \in deployed\\
			response!~=~'Sensor~deployed'\\
		\end{schema}
		
		
		
		\begin{schema}{LocationAlreadyCovered}
			\Xi TempMonitor\\
			location? : LOCATION\_TYPE\\
			response! : MESSAGE
			\where
			location? \in \ran map\\
			response!~=~'Location~already~covered'
		\end{schema}
		
		
		
		\begin{schema}{LocationUnknown}
			\Xi TempMonitor\\
			location? : LOCATION\_TYPE\\
			response! : MESSAGE
			\where
			location? \notin \ran map\\
			response!~=~'Location~not~covered'
		\end{schema}



  	    \begin{schema}{SensorUnknown}
			\Xi TempMonitor\\
			sensor? : SENSOR\_TYPE\\
			response! : MESSAGE
			\where
			sensor? \notin sensorRegistry \\
			response!~=~'Sensor~does~not~exist'
		\end{schema}

        \begin{schema}{SensorNotDeployed}
			\Xi TempMonitor\\
			sensor? : SENSOR\_TYPE\\
			response! : MESSAGE
			\where
			sensor? \notin deployed \\
			response!~=~'Sensor~not deployed'
		\end{schema}
		
		
		\subsection{Operations}
		
		\[ DeploySensor~\hat{=}~\\
		~~~(DeploySensorOK \wedge Success)~ \oplus\\
		~~~(SensorAlreadyDeployed \vee LocationAlreadyCovered) \]
		
		
		
		\[ ReadTemperature~\hat{=}~\\
		~~~(ReadTemperatureOK \wedge Success) \oplus LocationUnknown \]


  
        \[ ReplaceSensor~\hat{=}~\\
		~~~(ReplaceSensorOK \wedge Success) \oplus (SensorNotDeployed) \]



        \[ RemoveSensor~\hat{=}~\\
        ~~~((RemoveDeployedSensorOK \wedge Success) \oplus (SensorNotDeployed)) \\ \t6 \oplus \\
        ~~~((RemoveSensorOK \wedge Success) \oplus (SensorUnknown) )
        \]



        %\[ RemoveDeployedSensor~\hat{=}~\\
		%~~~(RemoveDeployedSensorOK \wedge Success) \oplus (SensorNotDeployed) \]
  
		
\end{spacing}
\end{document}
